\chapter{Classification System}
The classification system is one of the most powerful features of \lang,
allowing precise control over the classification and conditional processing of
large sets of data, whether it be external input or values generated from within
\lang\ itself. Virtually every conditional calculation is best represented
through use of the classification system.


\section{Classification Matcher}
Data classification is performed by the classification matcher (sometimes
referred to simply as the ``matcher''). Put simply, it is a function (defined by
\aref{cmatch}) that, given a vector of inputs, produces a boolean vector (which
may itself contain boolean vectors) determining if the given input conforms to a
set of stated rules. A set of rules operating on a set input vectors is
collectively known as a \term{classification}. The system that performs matching
based on classifications is referred to as a \term{classifier}.

A single classification can be separated into a set of rules, often referred to
as \term{matches} within the context of \lang. A single rule attempts to match
on a vector of inputs.\footnote{Scalar inputs are a special condition defined in
\sref{cmatch-scalar}.} A simple example of such a match is shown in
\fref{cmatch-ex-single}.

\begin{figure}[h]
  $$
    I = \left[
      \begin{array}{c}
        1 \\ 3 \\ 4 \\ 1
      \end{array}
    \right]
    \qquad
    M = \left[
      \begin{array}{c}
        1 \\ 4
      \end{array}
    \right]
    \quad
    \to
    \quad
    R = \left[
      \begin{array}{c}
        \top \\ \bot \\ \top \\ \top
      \end{array}
    \right].
  $$

  \caption{A simple classification match $M$ on input $I$ and its result vector
  $R$.}
  \label{f:cmatch-ex-single}
\end{figure}

In \fref{cmatch-ex-single}, the input vector $I$ is \term{matched} against the
rule $M$. The output is a boolean result vector $R$ which can be summarized with
the following rule:

$$
  R_n = \exists m\in M(m = I_n).
$$
\noindent
In other words, $R_n$ is $\top$ if $I_n\in M$ and is $\bot$ if $I_n\notin M$.
Under this definition, $M$ can be considered to be the \term{domain} under which
a given input $I_n$ is considered to be valid (a \term{match}).

We say that a classification rule \term{matches} if \emph{any} input matches.
That is:

$$
  \left[\textrm{The rule $M$ matches input $I$}\right]
  \iff
  \top\in R
$$
\noindent
Another way to think of this concept is the reduction of the result vector $R$
using a logical OR. Alternatively, one could assert that:

$$
  \left[\textrm{The rule $M$ matches input $I$}\right]
  \iff
  \sum\limits_n R_n \geq 1, \qquad R \in \set{0,1},
$$
\noindent
if an implementation were willing to use the sets \boolset and \set{1,0}
interchangeably.\footnote{See \sref{cmatch-int}.}

The following sections, however, serve to demonstrate that such a simple view of
the classification system, while useful for an introductory demonstration, is
not sufficient when considering the level of flexibility that is necessary to
handle more complicated data (in particular, when $I$ is a
matrix).\footnote{See $\Omega$-reductions, introduced in
\asref{cmatch}{omega-reduce}.}

%TODO: More example sections


\subsection{Classification Match (cmatch) Algorithm}
\label{a:cmatch}

The classification match (``cmatch'') algorithm is used to determine if a given
set of data matches a given set of classification criteria.

Let $I$ be the vector of input values.\footnote{$I$ may be a matrix (a vector
of vectors).} Let $M$ be the vector of predicates to match against $I$ such
that a match will be considered successful if \emph{any} predicate is true.
Since $I$ shall always be a vector of values---even if the vector contains only
one element (see algorithm below for comments on scalar values)---$M$ should be
a vector of one element if the desire is to match against a scalar value (rather
than a vector of values). Let $c$ (clear) be a boolean value\footnote{$1$ or $0$
when used within an integer context within the algorithm.} representing whether
the results of this operation should be logically AND'd together with the
prior cmatch result ($R'$ in the algorithm below); otherwise, the results will
be OR'd (see step \ref{a:cmatch-c} below).

Let $A\!\left(M,I,c,R'\right)$ (the ``algorithm'') be defined as:

\begin{enumerate}
  \item
  Let $R$ be the result vector.

  \item\label{a:cmatch-scalar}
  If the given input vector $I$ is a scalar, it should be converted to a vector
  of length 1 with the value of the single element being the original scalar
  value of $I$---that is, let $s$ be the original scalar value of $I$; then: $I
  = \left[ s \right]$. If $s$ is undefined, then an empty result vector should
  be returned.

  \item\label{a:cmatch:input-vectorize}
  Step \ref{a:cmatch-scalar} should also be done to the match vector $M$,
  yielding $M = \left[ s \right]$ where $s$ is the original scalar $M$. If $s$
  is undefined, then it should be treated as if it were the integer
  $0$.\footnote{Consistent with the behavior of the remainder of the DSL.}

  \item
  Step \ref{a:cmatch-scalar} should also be done to the prior result vector
  $R'$, yielding $R = \left[ s \right]$ where $s$ is the original scalar $R'$.
  This situation may result from recursing at step \ref{a:cmatch-mrecurse} when
  $R'_k$ is a scalar. If $s$ is undefined, then $R'$ should be initialized to an
  empty vector, implying a fresh match (no prior results).
  \goodbreak

  \item\label{a:cmatch-iter}
  The length of the result vector $R$~($\#R$) shall be the larger of the length
  of the input vector $I$~($\#I$) or the prior result vector $R'$~($\#R'$).
  For each $I_k \in I$:

  \begin{enumerate}
    \item\label{a:cmatch-mrecurse}
    If $I_k$ is a vector, recurse, beginning at step 1. Let $r =
    A(M_k,I_k,c,R'_k)$.

    \begin{align*}
      u &= \left\{
        \begin{array}{ll}
          \bot & \textrm{if }\#R' > 0, \\
          c    & \textrm{otherwise.}
        \end{array}
      \right. \\
      %
      R_k &= \left\{
        \begin{array}{ll}
          r & \textrm{if $R'_k$ is a vector or undefined}, \\
          \Omega(r,u) & \textrm{otherwise}.\footnotemark
        \end{array}
      \right.
    \end{align*}

    \footnotetext{\label{a:cmatch-order} If $R'_k$ is a scalar, we must ensure
    consistency with step \ref{a:cmatch-c} to ensure that the algorithm is not
    dependent on input or execution order. Note the use of $u$ in place of
    $c$---this ensures that, if there are any $R'$, we are consistent with the
    effects of step \ref{a:cmatch:fill} (but in reverse).}

    Continue with the next $I$ at step \ref{a:cmatch-iter}.

    \item
    \label{a:cmatch:omega-reduce}
    Otherwise, $I_k$ is a scalar. Let $t$ be a temporary (intermediate) scalar
    such that $t = \exists m \in M m(I_k)$.

    \item\label{a:cmatch-c}
    Let $v = \Omega\left(R'_k,c\right)$ and let
    $$
      R_k = \left\{
        \begin{array}{ll}
          v \wedge t & c = \top, \\
          v \vee   t & c = \bot.
        \end{array}
      \right.,
    $$

    where\footnote{$\Omega$ is simply the recursive reduction of a vector using
    a logical OR. This function exists to resolve the situation where $R'_k$ is
    a vector of values when $I_k$ is a scalar, which will occur when $M_k$ is
    scalar for any $k$ during one application of the cmatch algorithm and $M_k$
    is a vector for another iteration, where $R'$ is the previous match using
    scalars. Note also that $X$, according to the recursion rule, may only be
    undefined on the first iteration (in effect initializing the value).}

    $$
      \Omega\left(X,u\right) = \left\{
        \begin{array}{ll}
          u & \textrm{if X is undefined,} \\
          X & \textrm{if X is a scalar,} \\
          \exists x\in X \Omega(x,u) & \textrm{otherwise.}
        \end{array}
      \right. \>
      \mbox{
        $X \in \left\{\textrm{undefined},\top,\bot\right\}$
        or a vector.
      }
    $$
  \end{enumerate}

  \item\label{a:cmatch:fill}
  Let $v = \Omega\left(R'_k,c\right) \wedge \neg c$. If $\#R' > \#I$,
  $$
    R_k = \left\{
      \begin{array}{ll}
        v & \exists n\in I(n\textrm{ is a scalar}), \\
        \left[v\right] & \textrm{otherwise.}\footnotemark
      \end{array}
    \right.
    k \in \left\{j : \#I \leq j < \#R' \right\}.
  $$

  \footnotetext{Note that step \ref{a:cmatch:fill} will produce results
  inconsistent with the recursive step \ref{a:cmatch-mrecurse} if there exists
  an $I_n$ that is a matrix; this algorithm is not designed to handle such
  scenarios.}
\end{enumerate}

Given a set of classification criteria $C$ such that $C_k = M$ for some integer
$k$ and some application of $A$, and a vectorized clear flag $c$ such that $c_k$
is associated with $C_k$, the final result $F(\#C-1)$ shall be defined as

$$
  F(k) = \left\{
    \begin{array}{ll}
      A\left(C_k,I_k,c_k\right) & \textrm{k = 0,} \\
      A\bigl(C_k,I_k,c_k,F\!\left(k-1\right)\bigr) & \textrm{otherwise.}
    \end{array}
  \right.
$$

The order of recursion on $F$ need not be right-to-left; $A$ is defined such
that it will produce the same result when applied in any order. This is
necessary since the input may be provided in any order.\footnote{Ibid,
\ref{a:cmatch-order}.}

\subsubsection{Boolean Classification Match}
\label{s:cmatch-boolean}
A scalar boolean classification match $b$ may be obtained simply as $b =
\Omega\left(F,\bot\right)$, where $F$ and $\Omega$ are defined in the algorithm
above. Consequently, note that an empty result set $F$ will be treated as
$\bot$, since index $0$ will be undefined.

\subsubsection{Match Vector}
$M$ is defined to be a vector of predicates which serve to {\sl match} against a
vector of input values. Most frequently, predicates will likely be against scalar
values. In such a case, an implementation may choose to forego function
application for performance reasons and instead match directly against the
scalar value. However, this document will consider scalar matches in the context
of predicates as functions. As such, if $M$ is a matrix, then the results are
implementation-defined (since the value does not make sense within the algorithm
as defined).

\subsubsection{Integer Results}
\label{s:cmatch-int}
$A$ defines $R$ to be a vector/matrix of boolean values. However, it may be
useful to use the cmatch results in calculations; as such, implementations that
make use of or produce cmatch results are required to do one or both of the
following where $b$ is a boolean scalar:

\begin{enumerate}
  \item
  Implicitly consider $b$ to be $\textrm{int}\!\left(b\right)$ when used in
  calculations, and/or

  \item
  Perform the implicit conversion before $R$ is returned from $A$,
\end{enumerate}

where the function {\sl int} is defined as

$$
  \textrm{int}(b) = \left\{
    \begin{array}{ll}
      1 & \textrm{if }b = \top, \\
      0 & \textrm{if }b = \bot.
    \end{array}
  \right.\qquad
  b \in \left\{\top,\bot\right\}.
$$


\subsection{Scalar Classification Matches}
\label{s:cmatch-scalar}
Implementations may find it convenient to support scalar inputs and scalar
classification matches to represent matching ``all'' indexes of a vector.
\aref{cmatch} defines both a classification match ($R$, and consequently $F$)
and an input ($I$) to be a vector, which is generally sufficient. However, in
the case where the number of indexes of the inputs and results of other matches
may be arbitrary, it may be useful to apply a certain classification across all
indexes, which cannot be done when $c = \top$ using \aref{cmatch}.

The onus of such a feature is on the implementation---it should flag such input
($I$) as a scalar, which is necessary since $I$ is unconditionally converted to
a vector by step \asref{cmatch}{input-vectorize}. If an implementation decides
to support scalar classification matches, \emph{it must conform to this
section}.  Let such a scalar flag be denoted $s_k \inbool$ respective to input
$I_k$.  Handling of both $F$ and $I$ is discussed in the sections that follow.

\subsubsection{Mixing Scalar And Vectorized Inputs}
\label{s:cmatch-scalar-mixed}
Under the condition that $\exists v\in s(v=\top)$, the compiler must:

\begingroup
  % this definition is local to this group
  \def\siset{k \in\set{j : s_j = \top}}

  \begin{enumerate}
    \item
    Reorder inputs $I$ such that each scalar input $I_k, \siset$ be applied
    after all non-scalar inputs have been matched using \aref{cmatch}.
    \begin{enumerate}
      \item
      Consequently (and contrary to what was mentioned in \aref{cmatch}),
      application order of $A$ with respect to inputs $I$ \emph{does} in fact
      matter and implementations should ensure that this restriction holds
      during runtime.
    \end{enumerate}

    \item
    Before application of a scalar input, the scalar $I_k$ should be vectorized
    according to the following rule:

    $$
      I'_{k,l} = I_k,
      \qquad \siset,
      \; 0 \leq l < \#R',
    $$

    where $R'$ is the value immediately before the application of $I_k$ as
    defined in \aref{cmatch}.

    \item
    Application of \aref{cmatch} should then proceed as normal, using $I'$ in
    place of $I$.
  \end{enumerate}
\endgroup

\subsubsection{Converting Vectorized Match To Scalar}
As defined by \aref{cmatch}, the result $R$ will always be a vector. An
implementation may \emph{only} convert a vectorized match to a scalar using the
method defined in this section under the condition that $\forall v\in
s(v=\top)$; otherwise, there will be a loss of data (due to the expansion rules
defined in \sref{cmatch-scalar-mixed}). The implementation also \emph{must not}
reduce the vectorized match to a scalar using $\Omega$. An implementation
\emph{may}, however, $\Omega$-reduce the match result $R$ into an
\emph{separate} value as mentioned in \sref{cmatch-boolean}.

Under the condition that $\forall v\in s(v=\top)$, the system may post-process
$F$ (as defined in \aref{cmatch}) such that

$$
  F' = F_0,
$$

and return $F'$ in place of $F$.

Note also that $F'$ may be fed back into \aref{cmatch} as an input and that the
results will be consistent and well-defined according to
\sref{cmatch-scalar-mixed} (and, consequently, this section).
