% The TAME Programming Language Living Document
%
%  Copyright (C) 2021 Ryan Specialty, LLC.
%
%  Licensed under the Creative Commons Attribution-ShareAlike 4.0
%  International License.
%%

\documentclass[draft,toc=index]{scrartcl}
\usepackage[draft=false]{scrlayer-scrpage}
\usepackage{tpl}

\title{The TAME Programming Language}
\subtitle{Design and Implementation (Living Document)}

\author{Mike Gerwitz}
\date{May 2021}% TODO dynamic

% Copyright notice for bottom of first page
\setlength\footheight{28pt}
\cfoot[%
  {\tiny Copyright \textcopyright{} 2021 Ryan Specialty, LLC.
  CC-BY-SA 4.0.}\\\ccbysa]{\thepage}

% Begin section numbering at 0 to emphasize that it's foundational material
% not directly related to TAME itself
\setcounter{section}{-1}

\makeindex

\begin{document}

\maketitle

\begin{abstract}
  \tame{} is The Algebraic Metalanguage, a programming language and
  collection of tools designed to aid in the development, understanding,
  and maintenance of systems performing numerous calculations on a
  complex graph of dependencies, conditions, and a large number of
  inputs.  \tame{} has existed for over a decade, and while its initial
  design was successful and still in active use today, it does suffer
  from inconsistencies and tradeoffs that introduce certain impediments
  to users of the language, and compromise future optimizations and
  language evolution.  It also lacks documentation not just of the
  language itself, but also of the underlying principles and
  implementation.

  This document is an attempt to formally consider certain parts of
  \tame{} as it undergoes redesign and reimplementation as part of the
  \tamer{} project.  It is considered a living document---it is not
  likely to ever be a finished work.
\end{abstract}


\tableofcontents

% The TAME Programming Language Notational Conventions
%
%  Copyright (C) 2021 Ryan Specialty Group, LLC.
%
%  Licensed under the Creative Commons Attribution-ShareAlike 4.0
%  International License.
%%

\section{Notational Conventions}
This section provides a fairly terse overview of the foundational
  mathematical concepts used in this paper.
While we try to reason about \tame{} in terms of algebra,
  first-order logic;
  and set theory;
    notation varies even within those branches.
To avoid ambiguity,
  especially while introducing our own notation,
  core operators and concepts are explicitly defined below.

This section begins its numbering at~0.
This is not only a hint that \tame{} (and this paper) use 0-indexing,
  but also because equations; definitions; theorems; corollaries; and the
  like are all numbered relative to their section.
When you see any of these prefixed with ``0.'',
  this sets those references aside as foundational mathematical concepts
    that are not part of the theory and operation of \tame{} itself.


\subsection{Propositional Logic}
\index{logic!propositional}
We reproduce here certain axioms and corollaries of propositional logic for
  convenience and to clarify our interpretation of certain concepts.
The use of the symbols $\land$, $\lor$, and~$\neg$ are standard.
\indexsym\infer{infer}
\index{infer (\ensuremath\infer)}
The symbol $\infer$ means ``infer''.
We use $\implies$ in place of $\rightarrow$ for implication,
  since the latter is used to denote the mapping of a domain to a codomain
  in reference to functions.
We further use $\equiv$ in place of $\leftrightarrow$ to represent material
  equivalence.

\indexsym\land{conjunction}
\index{conjunction (\ensuremath{\land})}
\begin{definition}[Logical Conjunction]
  $p,q \infer (p\land q)$.
\end{definition}

\indexsym\lor{disjunction}
\index{disjunction (\ensuremath{\lor})}
\begin{definition}[Logical Disjunction]
  $p \infer (p\lor q)$ and $q \infer (p\lor q)$.
\end{definition}

\begin{definition}[$\land$-Associativity]\dfnlabel{conj-assoc}
  $(p \land (q \land r)) \infer ((p \land q) \land r)$.
\end{definition}

\begin{definition}[$\lor$-Associativity]\dfnlabel{disj-assoc}
  $(p \lor (q \lor r)) \infer ((p \lor q) \lor r)$.
\end{definition}

\begin{definition}[$\land$-Commutativity]\dfnlabel{conj-commut}
  $(p \land q) \infer (q \land p)$.
\end{definition}

\begin{definition}[$\lor$-Commutativity]\dfnlabel{disj-commut}
  $(p \lor q) \infer (q \lor p)$.
\end{definition}

\begin{definition}[$\land$-Simplification]\dfnlabel{conj-simpl}
  $p \land q \infer p$.
\end{definition}

\begin{definition}[Double Negation]\dfnlabel{double-neg}
  $\neg\neg p \infer p$.
\end{definition}

\indexsym\neg{negation}
\index{negation (\ensuremath{\neg})}
\index{law of excluded middle}
\begin{definition}[Law of Excluded Middle]
  $\infer (p \lor \neg p)$.
\end{definition}

\index{law of non-contradiction}
\begin{definition}[Law of Non-Contradiction]
  $\infer \neg(p \land \neg p)$.
\end{definition}

\index{De Morgan's theorem}
\begin{definition}[De Morgan's Theorem]\dfnlabel{demorgan}
  $\neg(p \land q) \infer (\neg p \lor \neg q)$
    and $\neg(p \lor q) \infer (\neg p \land \neg q)$.
\end{definition}

\indexsym\equiv{equivalence}
\index{equivalence!material (\ensuremath{\equiv})}
\begin{definition}[Material Equivalence]
  $p\equiv q \infer \big((p \land q) \lor (\neg p \land \neg q)\big)$.
\end{definition}

$\equiv$ denotes a logical identity.
Consequently,
  it'll often be used as a definition operator.

\indexsym{\!\!\implies\!\!}{implication}
\index{implication (\ensuremath{\implies})}
\begin{definition}[Implication]
  $p\implies q \infer (\neg p \lor q)$.
\end{definition}

\begin{definition}[Tautologies]\dfnlabel{prop-taut}
  $p\equiv (p\land p)$ and $p\equiv (p\lor p)$.
\end{definition}

\indexsym{\true}{boolean, true}
\indexsym{\false}{boolean, false}
\index{boolean!FALSE@\tamefalse{} (\false)}%
\index{boolean!TRUE@\tametrue{} (\true)}%
\begin{definition}[Truth Values]\dfnlabel{truth-values}
  $\infer\true$ and $\infer\neg\false$.
\end{definition}

\indexsym\Int{integer}
\index{integer (\Int)}%
\begin{definition}[Boolean/Integer Equivalency]\dfnlabel{bool-int}
  $\Set{0,1}\in\Int, \false \equiv 0$ and $\true \equiv 1$.
\end{definition}


\subsection{First-Order Logic and Set Theory}
\index{logic!first-order}
\indexsym\emptyset{set empty}
\indexsym{\Set{}}{set}
\index{set!empty (\ensuremath{\emptyset, \{\}})}
The symbol $\emptyset$ represents the empty set---%
  the set of zero elements.
We assume that the axioms of ZFC~set theory hold,
  but define $\in$ here for clarity.

\todo{Introduce set-builder notation, $\union$, $\intersect$.}
\indexsym\in{set membership}
\indexsym\union{set, union}
\indexsym\intersect{set, intersection}
\index{set!membership@membership (\ensuremath\in)}
\index{set!union (\ensuremath\union)}
\index{set!intersection (\ensuremath\intersect)}
\begin{definition}[Set Membership]
  $x \in S \equiv \Set{x} \intersect S \not= \emptyset.$
\end{definition}

\index{domain of discourse}
$\forall$ denotes first-order universal quantification (``for all''),
  and $\exists$ first-order existential quantification (``there exists''),
  over some domain of discourse.

\indexsym\exists{quantification, existential}
\index{quantification!existential (\ensuremath\exists)}
\begin{definition}[Existential Quantification]\dfnlabel{exists}
  $\Exists{x\in X}{P(x)} \equiv
    \true \in \Set{P(x) \mid x\in X}$.
\end{definition}

\indexsym\forall{quantification, universal}
\index{quantification!universal (\ensuremath\forall)}
\begin{definition}[Universal Quantification]\dfnlabel{forall}
  $\Forall{x\in X}{P(x)} \equiv \neg\Exists{x\in X}{\neg P(x)}$.
\end{definition}

\index{quantification!vacuous truth}
\begin{remark}[Vacuous Truth]\remlabel{vacuous-truth}
  By \dfnref{exists}, $\Exists{x\in\emptyset}P \equiv \false$
    and by \dfnref{forall}, $\Forall{x\in\emptyset}P \equiv \true$.
  And so we also have the tautologies $\infer \neg\Exists{x\in\emptyset}P$
    and $\infer \Forall{x\in\emptyset}P$.
  Empty domains lead to undesirable consequences---%
    in particular,
      we must carefully guard against them in \dfnref{quant-conn} and
        \dfnref{quant-elim} to maintain soundness.
\end{remark}

We also have this shorthand notation:

\index{quantification!\ensuremath{\forall x,y,z}}
\index{quantification!\ensuremath{\exists x,y,z}}
\begin{align}
  \Forall{x,y,z\in S}P \equiv
    \Forall{x\in S}{\Forall{y\in S}{\Forall{z\in S}P}}, \\
  \Exists{x,y,z\in S}P \equiv
    \Exists{x\in S}{\Exists{y\in S}{\Exists{z\in S}P}}.
\end{align}

\begin{definition}[Quantifiers Over Connectives]\dfnlabel{quant-conn}
  Assuming that $x$ is not free in $\varphi$,
  \begin{alignat*}{3}
    \varphi\land\Exists{x\in X}{P(x)}
      &\equiv \Exists{x\in X}{\varphi\land P(x)}, \\
    \varphi\lor\Exists{x\in X}{P(x)}
      &\equiv \Exists{x\in X}{\varphi\lor P(x)}
      \qquad&&\text{assuming $X\neq\emptyset$}.
  \end{alignat*}
\end{definition}

\begin{definition}[Quantifier Elimination]\dfnlabel{quant-elim}
  $\Exists{x\in X}{\varphi} \equiv \varphi$ assuming $X\neq\emptyset$
    and $x$ is not free in~$\varphi$.
\end{definition}


\subsection{Functions}
\indexsym{f, g}{function}
\indexsym\mapsto{function, map}
\indexsym\rightarrow{function, domain map}
\index{function}
\index{function!map (\ensuremath\mapsto)}
\index{map|see {function}}
\index{function!domain}
\index{function!codomain}
\index{domain|see {function, domain}}
\index{function!domain map (\ensuremath\rightarrow)}
The notation $f = x \mapsto x' : A\rightarrow B$ represents a function~$f$
  that maps from~$x$ to~$x'$,
    where $x\in A$ (the domain of~$f$) and $x'\in B$ (the co-domain of~$f$).

\indexsym\times{set, Cartesian product}
\index{set!Cartesian product (\ensuremath\times)}
A function $A\rightarrow B$ can be represented as the Cartesian
  product of its domain and codomain, $A\times B$.
For example,
  $x\mapsto x^2 : \Int\rightarrow\Int$ is represented by the set of ordered
  pairs $\Set{(x,x^2) \mid x\in\Int}$, which looks something like

\begin{equation*}
  \Set{\ldots,\,(0,0),\,(1,1),\,(2,4),\,(3,9),\,\ldots}.
\end{equation*}

\indexsym{[\,]}{function, image}
\index{function!image (\ensuremath{[\,]})}
\index{function!as a set}
The set of values over which some function~$f$ ranges is its \dfn{image},
  which is a subset of its codomain.
In the example above,
  both the domain and codomain are the set of integers~$\Int$,
  but the image is $\Set{x^2 \mid x\in\Int}$,
    which is clearly a subset of~$\Int$.

We therefore have

\begin{align}
  A \rightarrow B &\subset A\times B, \\
  f : A \rightarrow B &\infer f \subset A\times B, \\
  f = \alpha \mapsto \alpha' : A \rightarrow B
                  &= \Set{(\alpha,\alpha')
                            \mid \alpha\in A \land \alpha'\in B}, \\
  f[D\subseteq A] &= \Set{f(\alpha) \mid \alpha\in D} \subset B, \\
  f[] &= f[A].
\end{align}

\indexsym{()}{tuple}
\index{tuple (\ensuremath{()})}
\index{relation|see {function}}
An ordered pair $(x,y)$ is also called a \dfn{$2$-tuple}.
Generally,
  an \dfn{$n$-tuple} is used to represent an $n$-ary function,
    where by convention we have $(x)=x$.
So $f(x,y) = f((x,y)) = x+y$.
If we let $t=(x,y)$,
  then we also have $f(x,y) = ft$,
    which we'll sometimes write as a subscript~$f_t$ where disambiguation is
      necessary and where parenthesis may add too much noise;
        this notation is especially well-suited to indexes,
          as in $f_1$.
Binary functions are often written using \dfn{infix} notation;
  for example, we have $x+y$ rather than $+(x,y)$.

\begin{equation}
  f_x = f(x) \in \Set{b \mid (x,b) \in f}
\end{equation}


\subsubsection{Binary Operations On Functions}
\indexsym{R}{relation}
Consider two unary functions $f$ and~$g$,
  and a binary relation~$R$.
\indexsym{\bicomp{R}}{function, binary composition}
\index{function!binary composition (\ensuremath{\bicomp{R}})}
We introduce a notation~$\bicomp R$ to denote the composition of a binary
  function with two unary functions.

\begin{align}
  f &: A \rightarrow B \\
  g &: A \rightarrow D \\
  R &: B\times D \rightarrow F \\
  f \bicomp{R} g &= \alpha \mapsto f_\alpha R g_\alpha : A \rightarrow F
\end{align}

\indexsym\circ{function, composition}
\index{function!composition (\ensuremath\circ)}
Note that $f$ and~$g$ must share the same domain~$A$.
In that sense,
  this is the mapping of the operation~$R$ over the domain~$A$.
This is analogous to unary function composition~$f\circ g$.

\index{function!constant}
A scalar value~$x$ can be mapped onto some function~$f$ using a constant
  function.
For example,
  consider adding some number~$x$ to each element in the image of~$f$:

\begin{equation*}
  f \bicomp+ (\_\mapsto x) = \alpha \mapsto f_\alpha + x.
\end{equation*}

\indexsym{\_}{variable, wildcard}
\index{variable!wildcard/hole (\ensuremath{\_})}
The symbol~$\_$ is used to denote a variable that matches anything but is
  never referenced,
    and is often referred to as a ``wildcard'' (since it matches anything)
    or a ``hole'' (since its value goes nowhere).

Note that we consider the bracket notation for the image of a function
  $(f:A\rightarrow B)[A]$ to itself be a binary function.
Given that, we have $f\bicomp{[]} = f\bicomp{[A]}$ for functions returning
  functions (such as vectors of vectors in \secref{vec}).


\subsection{Monoids and Sequences}\seclabel{monoids}
\index{abstract algebra!monoid}
\index{monoid|see abstract algebra, monoid}
\begin{definition}[Monoid]\dfnlabel{monoid}
  Let $S$ be some set.  A \dfn{monoid} is a triple $\Monoid S\bullet e$
    with the axioms

  \begin{align}
    \bullet &: S\times S \rightarrow S
      \tag{Monoid Binary Closure} \\
    \Forall{a,b,c\in S&}{a\bullet(b\bullet c) = (a\bullet b)\bullet c)},
      \tag{Monoid Associativity} \\
    \Exists{e\in S&}{\Forall{a\in S}{e\bullet a = a\bullet e = a}}.
      \tag{Monoid Identity}\label{eq:monoid-identity}
  \end{align}
\end{definition}

\index{abstract algebra}
\index{abstract algebra!semigroup}
Monoids originate from abstract algebra.
A monoid is a semigroup with an added identity element~$e$.
Only the identity element must be commutative,
  but if the binary operation~$\bullet$ is \emph{also} commutative,
    then the monoid is a \dfn{commutative monoid}.\footnote{%
      A commutative monoid is less frequently referred to as an
        \dfn{abelian monoid},
          related to the common term \dfn{abelian group}.}

Consider some sequence of operations
  $x_0 \bullet\cdots\bullet x_n \in S$.
Intuitively,
  a monoid tells us how to combine that sequence into a single element
  of~$S$.
When the sequence has one or zero elements,
  we then use the identity element $e\in S$:
    as $x_0 \bullet e = x_0$ in the case of one element
    or $e \bullet e = e$ in the case of zero.

\indexsym\cdots{sequence}
\index{sequence}
\begin{definition}[Monoidic Sequence]\dfnlabel{monoid-seq}
Generally,
  given some monoid $\Monoid S\bullet e$ and a sequence $\Fam{x}jJ\in S$
  where $n<|J|$,
    we have
    $x_0\bullet x_1\bullet\cdots\bullet x_{n-1}\bullet x_n$
    represent the successive binary operation on all indexed elements
    of~$x$.
When it's clear from context that the index is increasing by a constant
  of~$1$,
    that notation is shortened to $x_0\bullet\cdots\bullet x_n$ to save
    space.
When $|J|=1$, then $n=0$ and we have the sequence $x_0$.
When $|J|=0$, then $n=-1$,
  and no such sequence exists,
  in which case we expand into the identity element~$e$.
\end{definition}

For example,
  given the monoid~$\Monoid\Int+0$,
  the sequence $1+2+\cdots+4+5$ can be shortened to
    $1+\cdots+5$ and represents the arithmetic progression
    $1+2+3+4+5=15$.
If $x=\Set{1,2,3,4,5}$,
  $x_0+\cdots+x_n$ represents the same sequence.
If $x=\Set{1}$,
  that sequence evaluates to $1=1$.
If $x=\Set{}$,
  we have $0$.

\index{conjunction!monoid}
\begin{lemma}\lemlabel{monoid-land}
  $\Monoid\Bool\land\true$ is a commutative monoid.
\end{lemma}
\begin{proof}
  $\Monoid\Bool\land\true$ is associative by \dfnref{conj-assoc}
    and commutative by \dfnref{conj-commut}.
  The identity element is~$\true\in\Bool$ by \dfnref{conj-simpl}.
\end{proof}

\index{disjunction!monoid}
\begin{lemma}\lemlabel{monoid-lor}
  $\Monoid\Bool\lor\false$ is a commutative monoid.
\end{lemma}
\begin{proof}
  $\Monoid\Bool\lor\false$ is associative by \dfnref{disj-assoc}
    and commutative by \dfnref{disj-commut}.
  The identity $\false\in\Bool$ follows from

  \begin{alignat*}{3}
    \false \lor p &\equiv p \lor \false &&\text{by \dfnref{disj-commut}} \\
                  &\equiv \neg(\neg p \land \neg\false)\qquad
                    &&\text{by \dfnref{demorgan}} \\
                  &\equiv \neg(\neg p) &&\text{by \dfnref{conj-simpl}} \\
                  &\equiv p. &&\text{by \dfnref{double-neg}} \tag*\qedhere
  \end{alignat*}
\end{proof}


\goodbreak% Fits well on its own page, if we're near a page boundary
\subsection{Vectors and Index Sets}\seclabel{vec}
\tame{} supports scalar, vector, and matrix values.
Unfortunately,
  its implementation history leaves those concepts a bit tortured.

A vector is a sequence of values, defined as a function of
  an index~set.

% TODO: font changes in index, making langle unavailable
%\indexsym{\Vector{}}{vector}
\index{vector!definition (\ensuremath{\Vector{}})}
\index{sequence|see vector}
\indexsym\Vectors{vector}
\index{real number (\ensuremath\Real)}
\indexsym\Real{real number}
\indexsym{\Fam{a}jJ}{index set}
\index{family|see {index set}}
\index{index set!_@notation (\ensuremath{\Fam{a}jJ})}
\begin{definition}[Vector]\dfnlabel{vec}
  Let $J\subset\Int$ represent an index set.
  A \dfn{vector}~$v\in\Vectors^\Real$ is a totally ordered sequence of
   elements represented as a function of an element of its index set:
  \begin{equation}\label{vec}
    v = \Vector{v_0,\ldots,v_j}^{\Real}_{j\in J}
      = j \mapsto v_j : J \rightarrow \Real.
  \end{equation}
\end{definition}

This definition means that $v_j = v(j)$,
  making the subscript a notational convenience.
We may omit the superscript such that $\Vectors^\Real=\Vectors$
  and $\Vector{\ldots}^\Real=\Vector{\ldots}$.

\index{vector!matrix}
\begin{definition}[Matrix]\dfnlabel{matrix}
  Let $J\subset\Int$ represent an index set.
  A \dfn{matrix}~$M\in\Matrices$ is a totally ordered sequence of
   elements represented as a function of an element of its index set:
  \begin{equation}
    M = \Vector{M_0,\ldots,M_j}^{\Vectors^\Real}_{j\in J}
      = j \mapsto M_j : J \rightarrow \Vectors^\Real.
  \end{equation}
\end{definition}

The consequences of \dfnref{matrix}---%
  defining a matrix as a vector of independent vectors---%
  are important.
This defines a matrix to be more like a multidimensional array,
  with no requirement that the lengths of the vectors be equal.

\begin{corollary}[Matrix Row Length Variance]\corlabel{matrix-row-len}
  $\infer \Exists{M\in\Matrices}{\neg\Forall*{j}{\Forall{k}{\len{M_j} = \len{M_k}}}}$.
\end{corollary}

\corref{matrix-row-len} can be read ``there exists some matrix~$M$ such that
  not all row lengths of~$M$ are equal''.
In other words---%
  the inner vectors of a matrix can vary in length.

Since a vector is a function,
  a vector or matrix can be converted into a set of unique elements like so:

\begin{alignat*}{2}
  \bigcup\Vector{\Vector{0,1},\Vector{2,2},\Vector{2,0}}\!\bicomp{[]}
    &\mapsto &&\bigcup\Vector{\Vector{0,1}\![],\Vector{2,2}\![],\Vector{2,0}[]}\![] \\
    &\mapsto &&\bigcup\Vector{\Set{0,1},\Set{2},\Set{2,0}}\![] \\
    &\mapsto &&\bigcup\Set{\Set{0,1},\Set{2},\Set{2,0}} \\
    &=       &&\Set{0,1,2}.
\end{alignat*}

% TODO: symbol does not render properly in index
\begin{definition}[Rank]\dfnlabel{rank}
\index{rank}
  The \dfn{rank} of some variable~$x$ is an integer value

  \begin{equation*}
    \|x\| =
      \begin{cases}
        2 &x\in\Matrices, \\
        1 &x\in\Vectors^\Real, \\
        0 &x\in\Real.
      \end{cases}
  \end{equation*}
\end{definition}

Intuitively, the rank represents the number of dimensions of some variable~$x$.
A scalar has zero dimensions (a point);
  a vector has one (a line);
  and a matrix has two (a plane).
\index{dimensions@see {rank}}
\index{rank!dimensions}
In \tame{},
  the rank is referred to as \dfn{dimensions} using the attribute
  \xmlattr{dim}.


\subsection{XML Notation}
\indexsym{\xml{<>}}{XML}
\index{XML!notation (\xml{<>})}
The grammar of \tame{} is XML.
Equivalence relations will be used to map source expressions to an
  underlying mathematical expression.
For example,

\begin{equation*}
  \xml{<foo bar="$x$" baz="$y$" />} \equiv x = y
\end{equation*}

\noindent
defines that pattern of \xmlnode{foo} expression to be materially
  equivalent to~$x=y$---%
    anywhere an equality relation appears,
      you could equivalently replace it with that XML representation without
        changing the meaning of the mathematical expression.

Variables may also bind to literals in an XML expression.
For example,

\begin{equation*}
  \xml{<quux $\alpha$ />},\qquad\alpha\in\Set{\emptystr, \xml{bar="baz"}}
\end{equation*}

\noindent
can represent either \xml{<quux />} or \xml{<quux bar="baz" />}.
\index{empty string}
$\emptystr$~represents the empty string.
Any text typeset in \texttt{typewriter} represents a literal
  string of characters.

% The TAME Programming Language Classification System
%
%  Copyright (C) 2021 Ryan Specialty Group, LLC.
%
%  Licensed under the Creative Commons Attribution-ShareAlike 4.0
%  International License.
%%

\section{Classification System}\seclabel{class}
\index{classification}
A \dfn{classification} is a user-defined abstraction that describes
  (``classifies'') arbitrary data.
Classifications can be used as predicates, generating functions, and can be
  composed into more complex classifications.
Nearly all conditions in \tame{} are specified using classifications.

\index{first-order logic!sentence}
\index{classification!coupling}
All classifications represent \dfn{first-order sentences}---%
  that is,
    they contain no \dfn{free variables}.
Intuitively,
  this means that all variables within a~classification are
  \dfn{tightly coupled} to the classification itself.
This limitation is mitigated through use of the template system.

\begin{axiom}[Classification Introduction]\axmlabel{class-intro}
\todo{Symbol in place of $=$ here ($\equiv$ not appropriate).}
\begin{alignat}{3}
    &\xml{<classify as="$c$" }&&\xml{yields="$\gamma$" desc}&&\xml{="$\_$"
          $\alpha$>}\label{eq:xml-classify} \\
    &\quad \MFam{M^0}jJkK   &&\VFam{v^0}jJ   &&\quad s^0    \nonumber\\[-4mm]
    &\quad \quad\vdots      &&\quad\vdots    &&\quad \vdots \nonumber\\
    &\quad \MFam{M^l}jJkK   &&\VFam{v^m}jJ   &&\quad s^n    \nonumber\\[-3mm]
    &\xml{</classify>}
      % NB: This -50mu needs adjustment if you change the alignment above!
      &&\mspace{-50mu}= \Classify^c_\gamma\left(\odot,M,v,s\right), \nonumber
\end{alignat}

\noindent
where

\begin{align}
    J &\subset\Int \neq\emptyset, \\
    \forall{j\in J}\Big(K_j &\subset\Int \neq\emptyset\Big), \\
    \forall{k}\Big(M^k &: J \rightarrow K_{j\in J} \rightarrow \Bool\Big),
                                            \label{eq:class-matrix} \\
    \forall{k}\Big(v^k &: J \rightarrow \Bool\Big), \\
    \forall{k}\Big(s^k &\in\Bool\Big), \\
    \alpha &\in\Set{\epsilon,\, \texttt{any="true"}}, \label{eq:xml-any-domain}
\end{align}

\noindent
and the monoid~$\odot$ is defined as

\begin{equation}\label{eq:classify-rel}
  \odot = \begin{cases}
        \Monoid\Bool\land\true &\alpha = \epsilon,\\
        \Monoid\Bool\lor\false &\alpha = \texttt{any="true"}.
      \end{cases}
\end{equation}
\end{axiom}


% This TODO was the initial motivation for this paper!
\todo{Emphasize index sets, both relationships and nonempty.}
We use a $4$-tuple $\Classify\left(\odot,M,v,s\right)$ to represent a
  $\odot_1$-classification
    (a classification with the binary operation $\land$ or~$\lor$)
  consisting of a combination of matrix~($M$), vector~($v$), and
    scalar~($s$) matches,
      rendered above in columns.\footnote{%
        The symbol~$\odot$ was chosen since the binary operation for a monoid
          is~$\bullet$
            (see \secref{monoids})
          and~$\odot$ looks vaguely like~$(\bullet)$,
            representing a portion of the monoid triple.}
A $\land$-classification is pronounced ``conjunctive classification'',
  and $\lor$ ``disjunctive''.\footnote{%
    Conjunctive and disjunctive classifications used to be referred to,
      respectively,
      as \dfn{universal} and \dfn{existential},
        referring to fact that
          $\forall\Set{a_0,\ldots,a_n}(a) \equiv a_0\land\ldots\land a_n$,
            and similarly for $\exists$.
    This terminology has changed since all classifications are in fact
      existential over their matches' index sets,
        and so the terminology would otherwise lead to confusion.}

The variables~$c$ and~$\gamma$ are required in~\tame{} but are both optional
  in our notation~$\Classify^c_\gamma$,
    and can be used to identify the two different data representations of
    the classification.\footnote{%
      \xpath{classify/@yields} is optional in the grammar of \tame{},
        but the compiler will generate one for us if one is not provided.
      As such,
        we will for simplicity consider it to be required here.}

$\alpha$~serves as a placeholder for an optional \xml{any="true"},
  with $\emptystr$~representing the empty string in~\eqref{eq:xml-any-domain}.
Note the wildcard variable matching \xmlattr{desc}---%
  its purpose is only to provide documentation.

\begin{corollary}[$\odot$ Commutative Monoid]\corlabel{odot-monoid}
  $\odot$ is a commutative monoid in \axmref{class-intro}.
\end{corollary}
\begin{proof}
  By \axmref{class-intro},
    $\odot$ must be a monoid.
  Assume $\alpha=\epsilon$.
  Then,
    $\odot = \Monoid\Bool\land\true$,
      which is proved by \lemref{monoid-land}.
  Next, assume $\alpha=\texttt{any="true"}$.
  Then,
    $\odot = \Monoid\Bool\lor\false$,
      which is proved by \lemref{monoid-land}.
\end{proof}

\index{classification!commutativity}
\index{compiler!classification commutativity}
While \axmref{class-intro} seems to imply an ordering to matches,
  users of the language are free to specify matches in any order
    and the compiler will rearrange matches as it sees fit.
This is due to the commutativity of~$\odot$ as proved by
  \corref{odot-monoid},
    and not only affords great ease of use to users of~\tame{},
      but also great flexibility to compiler writers.

For notational convenience,
  we will let

\begin{align}
  \odot^\land &= \Monoid\Bool\land\true, \\
  \odot^\lor &= \Monoid\Bool\lor\false.
\end{align}


\def\cpredmatseq{{M^0_j}_k \bullet\cdots\bullet {M^l_j}_k}
\def\cpredvecseq{v^0_j\bullet\cdots\bullet v^m_j}
\def\cpredscalarseq{s^0\bullet\cdots\bullet s^n}


\begin{axiom}[Classification-Predicate Equivalence]\axmlabel{class-pred}
  Let $\Classify^c_\gamma\left(\Monoid\Bool\bullet e,M,v,s\right)$ be a
    classification by~\axmref{class-intro}.
  We then have the first-order sentence
  \begin{equation*}
    c \equiv
      {} \Exists{j\in J}{\Exists{k\in K_j}\cpredmatseq\bullet\cpredvecseq}
        \bullet\cpredscalarseq.
  \end{equation*}
\end{axiom}


\begin{axiom}[Classification Yield]\axmlabel{class-yield}
  Let $\Classify^c_\gamma\left(\Monoid\Bool\bullet e,M,v,s\right)$ be a
    classification by~\axmref{class-intro}.
  Then,

  \begin{align}
    r &= \begin{cases}
           2 &M\neq\emptyset, \\
           1 &M=\emptyset \land v\neq\emptyset, \\
           0 &M\union v = \emptyset,
         \end{cases} \\
    \exists{j\in J}\Big(\exists{k\in K_j}\Big(
      \Gamma^2_{j_k} &= \cpredmatseq\bullet\cpredvecseq\bullet\cpredscalarseq
    \Big)\Big), \\
    %
    \exists{j\in J}\Big(
      \Gamma^1_j &= \cpredvecseq\bullet\cpredscalarseq
    \Big), \\
    %
    \Gamma^0 &= \cpredscalarseq. \\
    %
    \gamma &= \Gamma^r.
  \end{align}
\end{axiom}

\begin{theorem}[Classification Composition]\thmlabel{class-compose}
  Classifications may be composed to create more complex classifications
    using the classification yield~$\gamma$ as in~\axmref{class-yield}.
  This interpretation is equivalent to \axmref{class-pred} by
  \begin{equation}
    c \equiv \Exists{j\in J}{
               \Exists{k\in K_j}{\Gamma^2_{j_k}}
               \bullet \Gamma^1_j
             }
             \bullet \Gamma^0.
  \end{equation}
\end{theorem}

\def\eejJ{\equiv \exists{j\in J}\Big(}

\begin{proof}
  Expanding each~$\Gamma$ in \axmref{class-yield},
    we have

  \begin{alignat*}{3}
    c &\eejJ\Exists{k\in K_j}{\Gamma^2_{j_k}}
              \bullet \Gamma^1_j
            \Big)
            \bullet \Gamma^0
        &&\text{by \axmref{class-yield}} \\
      %
      &\eejJ\exists{k\in K_j}\Big(
              \cpredmatseq \bullet \cpredvecseq \bullet \cpredscalarseq
            \Big) \\
      &\hphantom{\eejJ}\;\cpredvecseq \bullet \cpredscalarseq \Big)
                \bullet \cpredscalarseq, \\
      %
      &\eejJ\exists{k\in K_j}\Big(\cpredmatseq\Big)
            \bullet \cpredvecseq \bullet \cpredscalarseq \\
      &\hphantom{\eejJ}\;\cpredvecseq \bullet \cpredscalarseq \Big)
                \bullet \cpredscalarseq,
        &&\text{by \dfnref{quant-conn}} \\
      %
      &\eejJ\exists{k\in K_j}\Big(\cpredmatseq\Big)
        &&\text{by \dfnref{prop-taut}} \\
      &\hphantom{\eejJ}\;\cpredvecseq \bullet \cpredscalarseq \Big)
                \bullet \cpredscalarseq, \\
      %
      &\eejJ\exists{k\in K_j}\Big(\cpredmatseq\Big)
        &&\text{by \dfnref{quant-conn}} \\
      &\hphantom{\eejJ}\;\cpredvecseq\Big) \bullet \cpredscalarseq
                \bullet \cpredscalarseq, \\
      %
      &\eejJ\exists{k\in K_j}\Big(\cpredmatseq\Big)
        &&\text{by \dfnref{prop-taut}} \\
      &\hphantom{\eejJ}\;\cpredvecseq\Big)
                \bullet \cpredscalarseq.
        \tag*{\qedhere} \\
  \end{alignat*}
\end{proof}


\begin{lemma}[Classification Predicate Vacuity]\lemlabel{class-pred-vacu}
  Let $\Classify^c_\gamma\left(\Monoid\Bool\bullet e,\emptyset,\emptyset,\emptyset\right)$
    be a classification by~\axmref{class-intro}.
  $\odot$ is a monoid by \corref{odot-monoid}.
  Then $c \equiv \gamma \equiv e$.
\end{lemma}
\begin{proof}
  First consider $c$.
  \begin{alignat}{3}
    c &\equiv \Exists{j\in J}{\Exists{k}{e}\bullet e} \bullet e
        \qquad&&\text{by \dfnref{monoid-seq}} \label{p:cri-c} \\
      &\equiv \Exists{j\in J}{e \bullet e} \bullet e
        &&\text{by \dfnref{quant-elim}} \\
      &\equiv \Exists{j\in J}{e} \bullet e
        &&\text{by \ref{eq:monoid-identity}} \\
      &\equiv e \bullet e
        &&\text{by \dfnref{quant-elim}} \\
      &\equiv e.
        &&\text{by \ref{eq:monoid-identity}}
  \end{alignat}

  For $\gamma$,
    we have $r=0$ by \axmref{class-yield},
    and so by similar steps as~$c$,
      $\gamma=\Gamma^r=e$.
  Therefore $c\equiv e$.
\end{proof}


\begin{figure}[h]\label{fig:always-never}
  \begin{alignat*}{3}
    \begin{aligned}
      \xml{<classify }&\xml{as="always" yields="alwaysTrue"} \xmlnl
                      &\xml{desc="Always true" />}
    \end{aligned}
      \quad&=\quad
      \Classify^\texttt{always}_\texttt{alwaysTrue}
        &&\left(\odot^\land,\emptyset,\emptyset,\emptyset\right). \\
    %
    \begin{aligned}
      \xml{<classify }&\xml{as="never" yields="neverTrue"} \xmlnl
                      &\xml{any="true"} \xmlnl
                      &\xml{desc="Never true" />}
    \end{aligned}
      \quad&=\quad
      \Classify^\texttt{never}_\texttt{neverTrue}
        &&\left(\odot^\lor,\emptyset,\emptyset,\emptyset\right).
  \end{alignat*}
  \caption{\tameclass{always} and \tameclass{never} from package
             \tamepkg{core/base}.}
\end{figure}

Figure~\ref{fig:always-never} demonstrates \lemref{class-pred-vacu} in the
  definitions of the classifications \tameclass{always} and
  \tameclass{never}.
These classifications are typically referenced directly for clarity rather
  than creating other vacuous classifications,
    encapsulating \lemref{class-pred-vacu}.


\begin{theorem}[Classification Rank Independence]\thmlabel{class-rank-indep}
  Let $\odot=\Monoid\Bool\bullet e$.
  Then,
  \begin{equation}
    \Classify_\gamma\left(\odot,M,v,s\right)
      \equiv \Classify\left(
        \odot,
        \Classify_{\gamma'''}\left(\odot,M,\emptyset,\emptyset\right),
        \Classify_{\gamma''}\left(\odot,\emptyset,v,\emptyset\right),
        \Classify_{\gamma'}\left(\odot,\emptyset,\emptyset,s\right)
      \right).
  \end{equation}
\end{theorem}

\begin{proof}
  First,
    by \axmref{class-yield},
    observe these special cases following from \lemref{class-pred-vacu}:

  \begin{alignat}{3}
    \Gamma'''^2 &= \cpredmatseq, \qquad&&\text{assuming $v\union s=\emptyset$} \\
    \Gamma''^1 &= \cpredvecseq,        &&\text{assuming $M\union s=\emptyset$}\\
    \Gamma'^0 &= \cpredscalarseq.      &&\text{assuming $M\union v=\emptyset$}
  \end{alignat}

  By \thmref{class-compose},
    we must prove

  \begin{align}
    \Exists{j\in J}{
        \Exists{k\in K_j}{\cpredmatseq}
        \bullet \cpredvecseq
      }
      \bullet \cpredscalarseq \nonumber\\
    \equiv c \equiv
    \Exists{j\in J}{
        \Exists{k\in K_j}{\gamma'''_{j_k}}
        \bullet \gamma''_j
      }
      \bullet \gamma'. \label{eq:rank-indep-goal}
  \end{align}

  By \axmref{class-yield},
    we have $r'''=2$, $r''=1$, and $r'=0$,
    and so $\gamma'''=\Gamma'''^2$,
      $\gamma''=\Gamma''^1$,
      and $\gamma'=\Gamma'^0$.
  By substituting these values in~\ref{eq:rank-indep-goal},
    the theorem is proved.
\end{proof}

\begin{corollary}[Classification As Proposition]
  Classifications with $M\union v=\emptyset$ or with constant index sets can
    be represented by propositional logic
      (that is---without first-order logic).
\end{corollary}
\begin{proof}
  Assume $M\union v=\emptyset$.
  By \thmref{class-rank-indep},

  \begin{align*}
    c &\equiv \cpredscalarseq,
  \end{align*}

  \noindent
  which is a propositional formula.

  Similarly,
    if we define our index set~$J$ to be constant
      (such that it is known at compile-time)\footnote{%
        Alternatively,
          we could set an upper bound for~$J$,
          always expand into that upper bound,
          and then let undefined values of $v^m_j$ be~$e$.},
        we are then able to eliminate existential quantification over~$J$
        as follows:
  Then,

  \begin{align}\label{eq:prop-vec}
    c &\equiv \Exists{j\in J}{\cpredvecseq}, \nonumber\\
      &\equiv \left(v^0_0\bullet\cdots\bullet v^m_0\right)
              \lor\cdots\lor
              \left(v^0_{|J|-1}\bullet\cdots\bullet v^m_{|J|-1}\right),
  \end{align}
  \noindent
  which is a propositional formula.

  Similarly,
    for matrices,

  \begin{align}
    c &\equiv \Exists{j\in J}{\Exists{k\in K_j}{\cpredmatseq}}, \nonumber\\
      &\equiv \Exists{j\in J}{
                \left({M^0_j}_0\bullet\cdots\bullet{M^0_j}_{|K_j|-1}\right)
                \lor\cdots\lor
                \left({M^l_j}_0\bullet\cdots\bullet{M^l_j}_{|K_j|-1}\right)
              },
  \end{align}
  \noindent
  and then proceed as in~\ref{eq:prop-vec}.
\end{proof}

\INCOMPLETE{The classification definitions are incomplete!}

These definitions may also be used as a form of pattern matching to look up
  a corresponding variable.
For example,
  if we have $\Classify^\texttt{foo}$ and want to know its \xmlattr{yields},
    we can write~$\Classify^\texttt{foo}_\gamma$ to bind the
    \xmlattr{yields} to~$\gamma$.\footnote{%
      This is conceptually like a symbol table lookup in the compiler.}

\mremark{Note that these illustrate \emph{scalar} values only.}
Consider the following classification $\Classify^\texttt{cost-exceeded}$.
Let~\tameparam{cost} be a scalar parameter.

\index{classification!classify@\xmlnode{classify}}
\begin{lstlisting}
  <classify as="cost-exceeded" desc="Cost of item is too expensive">
    <t:match-gt on="cost" value="100.00" />
  </classify>
\end{lstlisting}

\noindent
is then equivalent to the proposition

\begin{equation*}
  \tameclass{cost-exceeded} \equiv \tameparam{cost} > 100.00.
\end{equation*}

\index{classification!domain}
A classification is either \true or~\false.
Let $\tameparam{cost}=150.00$.
Then,

\begin{align*}
  \tameclass{cost-exceeded} & \equiv \tameparam{cost} > 100.00 \\
                            & \equiv 150.00 > 100.00 \\
                            & \equiv \true.
\end{align*}

Each \xmlnode{match} of a classification is a~\dfn{predicate}.
Multiple predicates are by default joined by conjunction:

\begin{lstlisting}
  <classify as="pool-hazard" desc="Hazardous pool">
    <match on="diving_board" />
    <t:match-lt on="pool_depth_ft" value="8" />
  </classify>
\end{lstlisting}

\noindent
is equivalent to the proposition

\begin{equation*}
  \tameclass{pool-hazard} \equiv \tameparam{diving\_board}
    \land \tameparam{pool\_depth\_ft} < 8.
\end{equation*}


\subsection{Matches}
\todo{Non-scalar values.}
\begin{definition}[Match Equality]
  \begin{equation*}
    \xml{<match on="$x$" value="$y$" />} \equiv x = y.
  \end{equation*}
\end{definition}

\begin{definition}[Match Equality Short Form]
  \begin{equation*}
    \xml{<match on="$x$" />}
      \equiv \xml{<match on="$x$" value="TRUE" />}.
  \end{equation*}
\end{definition}

\begin{definition}[Match Equality Long Form]
  \begin{alignat*}{2}
    \xml{<match on="$x$" value="$y$" />}
      &\equiv {}&&\xml{<match on="$x$">} \\
      &         &&\quad \xml{<c:eq>} \\
      &         &&\quad\quad \xml{<c:value-of name="$y$">} \\
      &         &&\quad \xml{</c:eq>} \\
      &         &&\xml{</match>} \\
      &\equiv {}&&\xml{<t:match-eq on="$x$" value="$y$" />}.
  \end{alignat*}
\end{definition}

\begin{definition}[Match Membership Equivalence]
  When $T$ is a type defined with \xmlnode{typedef},

  \begin{equation*}
    \xml{<match on="$x$" anyOf="$T$" />} \equiv x \in T.
  \end{equation*}
\end{definition}



% Appendix may be enabled with `./configure --enable-appendix'.
\iftplappendix
 \clearpage
 \appendix
 
\section{Meta: Typesetting}
This appendix is a meta-document describing typographic considerations for
  this document.
It is intended to be included in debug/developer builds,
  not to be included as official documentation for the software.

\subsection{$\Classify$}
\index{classification!\ensuremath{\Classify} design}
The symbol representing classification is defined in \secref{class}.
It uses the capital Fraktur letter~`C',
  typeset as~$\Classify\!\!$.
Compare this side-by-side with the summation operator Sigma:

\def\EXSUM{\sum_k^n k}
\def\EXCLASS{\Classify^\texttt{as}_\texttt{yields} P}

\begin{equation*}
 \EXSUM \mspace{50mu} \EXCLASS.
\end{equation*}

These are both written inline, respectively, as
  $\EXSUM$ and $\EXCLASS$.

Classifications are canonically referred to by their \xmlattr{as}~name,
  which makes for a bit of an awkward-looking construction when the
  superscript is provided but not the subscript.
For example, consider the classification $\Classify^\texttt{foo-bar}$ inline,
  compared to~$\Classify_\texttt{foo-bar}$.
Now compare the display style

\begin{equation*}
  \Classify^\texttt{foo-bar}
  \mspace{25mu}\text{vs.}\mspace{25mu}
  \Classify_\texttt{foo-bar}.
\end{equation*}

Of course,
  referring to the \xmlattr{yields}~name will use the subscript,
    which (at least to the author) feels more natural.
Why not swap them, then?

The superscript always denotes a scalar Boolean value.
The subscript,
  however,
  has a more complex type that's dependent on the predicates of the classification.
Let's say we wanted to denote a classification with a dimensionality of~$2$:
  $\Classify^\texttt{foo-bar}_{\texttt{fooBar}^2}$ versus
  $\Classify_\texttt{foo-bar}^{\texttt{fooBar}^2}$,
    typeset in display style as

\begin{equation*}
  \Classify^\texttt{foo-bar}_{\texttt{fooBar}^2}
  \mspace{25mu}\text{vs.}\mspace{25mu}
  \Classify_\texttt{foo-bar}^{\texttt{fooBar}^2}.
\end{equation*}

The amount of vertical space taken up by the first style is unchanged by the
  superscript on the subscript,
    but that's not true of the second style.

The final consideration is that the subscript of the summation,
  when the superscript is omitted,
  denotes the range or set of values for the sum.
For example,
  one may have $\sum_{0\leq k \leq n}$ or
  $\sum_{a\in A}$.
Having the subscript of $\Classify$ represent the more complex set of values
  is therefore more analogous to the sum,
    and better fits readers' intuitive notational expectations.

It is also worth noting that the distinction between the two is historical---%
  \xmlattr{as} used to represent an accumulator,
    which is a long-removed feature;
      references to~\xmlattr{as} in \tame{} today end up resolving
      to~\xmlattr{yields} anyway.
If that name is repurposed,
  one potential option is to have it take the place of~\xmlattr{yields},
    in which case the superscript in~$\Classify$ goes away and the
      notational awkwardness is removed.

A final note on the choice of character:
Admittedly, $\Classify$ does look a bit threatening,
  but one could also interpret it as ``bold and distinguished''.
$\mathcal{C}$ was considered,
  but it looks too childish (and perhaps Comic Sans-like) when typeset large:

\begin{equation*}
  \displaystyle\mathop{\hbox{\huge$\mathcal{C}$}}^\texttt{rejected}_\texttt{style}.
\end{equation*}

It's easy enough to change in the future if we change our minds.

\fi

% Ensure Copyright line does not show
\cfoot[\thepage]{\thepage}

\printbibliography[heading=bibintoc]

\clearpage
\printindex

\end{document}

