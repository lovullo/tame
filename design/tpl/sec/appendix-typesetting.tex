
\section{Meta: Typesetting}
This appendix is a meta-document describing typographic considerations for
  this document.
It is intended to be included in debug/developer builds,
  not to be included as official documentation for the software.

\subsection{$\Classify$}
\index{classification!\ensuremath{\Classify} design}
The symbol representing classification is defined in \secref{class}.
It uses the capital Fraktur letter~`C',
  typeset as~$\Classify\!\!$.
Compare this side-by-side with the summation operator Sigma:

\def\EXSUM{\sum_k^n k}
\def\EXCLASS{\Classify^\texttt{as}_\texttt{yields} P}

\begin{equation*}
 \EXSUM \mspace{50mu} \EXCLASS.
\end{equation*}

These are both written inline, respectively, as
  $\EXSUM$ and $\EXCLASS$.

Classifications are canonically referred to by their \xmlattr{as}~name,
  which makes for a bit of an awkward-looking construction when the
  superscript is provided but not the subscript.
For example, consider the classification $\Classify^\texttt{foo-bar}$ inline,
  compared to~$\Classify_\texttt{foo-bar}$.
Now compare the display style

\begin{equation*}
  \Classify^\texttt{foo-bar}
  \mspace{25mu}\text{vs.}\mspace{25mu}
  \Classify_\texttt{foo-bar}.
\end{equation*}

Of course,
  referring to the \xmlattr{yields}~name will use the subscript,
    which (at least to the author) feels more natural.
Why not swap them, then?

The superscript always denotes a scalar Boolean value.
The subscript,
  however,
  has a more complex type that's dependent on the predicates of the classification.
Let's say we wanted to denote a classification with a dimensionality of~$2$:
  $\Classify^\texttt{foo-bar}_{\texttt{fooBar}^2}$ versus
  $\Classify_\texttt{foo-bar}^{\texttt{fooBar}^2}$,
    typeset in display style as

\begin{equation*}
  \Classify^\texttt{foo-bar}_{\texttt{fooBar}^2}
  \mspace{25mu}\text{vs.}\mspace{25mu}
  \Classify_\texttt{foo-bar}^{\texttt{fooBar}^2}.
\end{equation*}

The amount of vertical space taken up by the first style is unchanged by the
  superscript on the subscript,
    but that's not true of the second style.

The final consideration is that the subscript of the summation,
  when the superscript is omitted,
  denotes the range or set of values for the sum.
For example,
  one may have $\sum_{0\leq k \leq n}$ or
  $\sum_{a\in A}$.
Having the subscript of $\Classify$ represent the more complex set of values
  is therefore more analogous to the sum,
    and better fits readers' intuitive notational expectations.

It is also worth noting that the distinction between the two is historical---%
  \xmlattr{as} used to represent an accumulator,
    which is a long-removed feature;
      references to~\xmlattr{as} in \tame{} today end up resolving
      to~\xmlattr{yields} anyway.
If that name is repurposed,
  one potential option is to have it take the place of~\xmlattr{yields},
    in which case the superscript in~$\Classify$ goes away and the
      notational awkwardness is removed.

A final note on the choice of character:
Admittedly, $\Classify$ does look a bit threatening,
  but one could also interpret it as ``bold and distinguished''.
$\mathcal{C}$ was considered,
  but it looks too childish (and perhaps Comic Sans-like) when typeset large:

\begin{equation*}
  \displaystyle\mathop{\hbox{\huge$\mathcal{C}$}}^\texttt{rejected}_\texttt{style}.
\end{equation*}

It's easy enough to change in the future if we change our minds.
