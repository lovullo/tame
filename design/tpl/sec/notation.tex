% The TAME Programming Language Notational Conventions
%
%  Copyright (C) 2021 Ryan Specialty Group, LLC.
%
%  Licensed under the Creative Commons Attribution-ShareAlike 4.0
%  International License.
%%

\section{Notational Conventions}
This section provides a fairly terse overview of the foundational
  mathematical concepts used in this paper.
While we try to reason about \tame{} in terms of algebra,
  first-order logic;
  and set theory;
    notation varies even within those branches.
To avoid ambiguity,
  especially while introducing our own notation,
  core operators and concepts are explicitly defined below.

This section begins its numbering at~0.
This is not only a hint that \tame{} (and this paper) use 0-indexing,
  but also because equations; definitions; theorems; corollaries; and the
  like are all numbered relative to their section.
When you see any of these prefixed with ``0.'',
  this sets those references aside as foundational mathematical concepts
    that are not part of the theory and operation of \tame{} itself.


\subsection{Propositional Logic}
\index{logic!propositional}
We reproduce here certain axioms and corollaries of propositional logic for
  convenience and to clarify our interpretation of certain concepts.
The use of the symbols $\land$, $\lor$, and~$\neg$ are standard.
\indexsym\infer{infer}
\index{infer (\ensuremath\infer)}
The symbol $\infer$ means ``infer''.
We use $\implies$ in place of $\rightarrow$ for implication,
  since the latter is used to denote the mapping of a domain to a codomain
  in reference to functions.
We further use $\equiv$ in place of $\leftrightarrow$ to represent material
  equivalence.

\indexsym\land{conjunction}
\index{conjunction (\ensuremath{\land})}
\begin{definition}[Logical Conjunction]
  $p,q \infer (p\land q)$.
\end{definition}

\indexsym\lor{disjunction}
\index{disjunction (\ensuremath{\lor})}
\begin{definition}[Logical Disjunction]
  $p \infer (p\lor q)$ and $q \infer (p\lor q)$.
\end{definition}

\begin{definition}[$\land$-Associativity]\dfnlabel{conj-assoc}
  $(p \land (q \land r)) \infer ((p \land q) \land r)$.
\end{definition}

\begin{definition}[$\lor$-Associativity]\dfnlabel{disj-assoc}
  $(p \lor (q \lor r)) \infer ((p \lor q) \lor r)$.
\end{definition}

\begin{definition}[$\land$-Commutativity]\dfnlabel{conj-commut}
  $(p \land q) \infer (q \land p)$.
\end{definition}

\begin{definition}[$\lor$-Commutativity]\dfnlabel{disj-commut}
  $(p \lor q) \infer (q \lor p)$.
\end{definition}

\begin{definition}[$\land$-Simplification]\dfnlabel{conj-simpl}
  $p \land q \infer p$.
\end{definition}

\begin{definition}[Double Negation]\dfnlabel{double-neg}
  $\neg\neg p \infer p$.
\end{definition}

\indexsym\neg{negation}
\index{negation (\ensuremath{\neg})}
\index{law of excluded middle}
\begin{definition}[Law of Excluded Middle]
  $\infer (p \lor \neg p)$.
\end{definition}

\index{law of non-contradiction}
\begin{definition}[Law of Non-Contradiction]
  $\infer \neg(p \land \neg p)$.
\end{definition}

\index{De Morgan's theorem}
\begin{definition}[De Morgan's Theorem]\dfnlabel{demorgan}
  $\neg(p \land q) \infer (\neg p \lor \neg q)$
    and $\neg(p \lor q) \infer (\neg p \land \neg q)$.
\end{definition}

\indexsym\equiv{equivalence}
\index{equivalence!material (\ensuremath{\equiv})}
\begin{definition}[Material Equivalence]
  $p\equiv q \infer \big((p \land q) \lor (\neg p \land \neg q)\big)$.
\end{definition}

$\equiv$ denotes a logical identity.
Consequently,
  it'll often be used as a definition operator.

\indexsym{\!\!\implies\!\!}{implication}
\index{implication (\ensuremath{\implies})}
\begin{definition}[Implication]
  $p\implies q \infer (\neg p \lor q)$.
\end{definition}

\begin{definition}[Tautologies]\dfnlabel{prop-taut}
  $p\equiv (p\land p)$ and $p\equiv (p\lor p)$.
\end{definition}

\indexsym{\true}{boolean, true}
\indexsym{\false}{boolean, false}
\index{boolean!FALSE@\tamefalse{} (\false)}%
\index{boolean!TRUE@\tametrue{} (\true)}%
\begin{definition}[Truth Values]\dfnlabel{truth-values}
  $\infer\true$ and $\infer\neg\false$.
\end{definition}

\indexsym\Int{integer}
\index{integer (\Int)}%
\begin{definition}[Boolean/Integer Equivalency]\dfnlabel{bool-int}
  $\Set{0,1}\in\Int, \false \equiv 0$ and $\true \equiv 1$.
\end{definition}


\subsection{First-Order Logic and Set Theory}
\index{logic!first-order}
\indexsym\emptyset{set empty}
\indexsym{\Set{}}{set}
\index{set!empty (\ensuremath{\emptyset, \{\}})}
The symbol $\emptyset$ represents the empty set---%
  the set of zero elements.
We assume that the axioms of ZFC~set theory hold,
  but define $\in$ here for clarity.

\todo{Introduce set-builder notation, $\union$, $\intersect$.}
\indexsym\in{set membership}
\indexsym\union{set, union}
\indexsym\intersect{set, intersection}
\index{set!membership@membership (\ensuremath\in)}
\index{set!union (\ensuremath\union)}
\index{set!intersection (\ensuremath\intersect)}
\begin{definition}[Set Membership]
  $x \in S \equiv \Set{x} \intersect S \not= \emptyset.$
\end{definition}

\index{domain of discourse}
$\forall$ denotes first-order universal quantification (``for all''),
  and $\exists$ first-order existential quantification (``there exists''),
  over some domain of discourse.

\indexsym\exists{quantification, existential}
\index{quantification!existential (\ensuremath\exists)}
\begin{definition}[Existential Quantification]\dfnlabel{exists}
  $\Exists{x\in X}{P(x)} \equiv
    \true \in \Set{P(x) \mid x\in X}$.
\end{definition}

\indexsym\forall{quantification, universal}
\index{quantification!universal (\ensuremath\forall)}
\begin{definition}[Universal Quantification]\dfnlabel{forall}
  $\Forall{x\in X}{P(x)} \equiv \neg\Exists{x\in X}{\neg P(x)}$.
\end{definition}

\index{quantification!vacuous truth}
\begin{remark}[Vacuous Truth]\remlabel{vacuous-truth}
  By \dfnref{exists}, $\Exists{x\in\emptyset}P \equiv \false$
    and by \dfnref{forall}, $\Forall{x\in\emptyset}P \equiv \true$.
  And so we also have the tautologies $\infer \neg\Exists{x\in\emptyset}P$
    and $\infer \Forall{x\in\emptyset}P$.
  Empty domains lead to undesirable consequences---%
    in particular,
      we must carefully guard against them in \dfnref{quant-conn} and
        \dfnref{quant-elim} to maintain soundness.
\end{remark}

We also have this shorthand notation:

\index{quantification!\ensuremath{\forall x,y,z}}
\index{quantification!\ensuremath{\exists x,y,z}}
\begin{align}
  \Forall{x,y,z\in S}P \equiv
    \Forall{x\in S}{\Forall{y\in S}{\Forall{z\in S}P}}, \\
  \Exists{x,y,z\in S}P \equiv
    \Exists{x\in S}{\Exists{y\in S}{\Exists{z\in S}P}}.
\end{align}

\begin{definition}[Quantifiers Over Connectives]\dfnlabel{quant-conn}
  Assuming that $x$ is not free in $\varphi$,
  \begin{alignat*}{3}
    \varphi\land\Exists{x\in X}{P(x)}
      &\equiv \Exists{x\in X}{\varphi\land P(x)}, \\
    \varphi\lor\Exists{x\in X}{P(x)}
      &\equiv \Exists{x\in X}{\varphi\lor P(x)}
      \qquad&&\text{assuming $X\neq\emptyset$}.
  \end{alignat*}
\end{definition}

\begin{definition}[Quantifier Elimination]\dfnlabel{quant-elim}
  $\Exists{x\in X}{\varphi} \equiv \varphi$ assuming $X\neq\emptyset$
    and $x$ is not free in~$\varphi$.
\end{definition}


\subsection{Functions}
\indexsym{f, g}{function}
\indexsym\mapsto{function, map}
\indexsym\rightarrow{function, domain map}
\index{function}
\index{function!map (\ensuremath\mapsto)}
\index{map|see {function}}
\index{function!domain}
\index{function!codomain}
\index{domain|see {function, domain}}
\index{function!domain map (\ensuremath\rightarrow)}
The notation $f = x \mapsto x' : A\rightarrow B$ represents a function~$f$
  that maps from~$x$ to~$x'$,
    where $x\in A$ (the domain of~$f$) and $x'\in B$ (the co-domain of~$f$).

\indexsym\times{set, Cartesian product}
\index{set!Cartesian product (\ensuremath\times)}
A function $A\rightarrow B$ can be represented as the Cartesian
  product of its domain and codomain, $A\times B$.
For example,
  $x\mapsto x^2 : \Int\rightarrow\Int$ is represented by the set of ordered
  pairs $\Set{(x,x^2) \mid x\in\Int}$, which looks something like

\begin{equation*}
  \Set{\ldots,\,(0,0),\,(1,1),\,(2,4),\,(3,9),\,\ldots}.
\end{equation*}

\indexsym{[\,]}{function, image}
\index{function!image (\ensuremath{[\,]})}
\index{function!as a set}
The set of values over which some function~$f$ ranges is its \dfn{image},
  which is a subset of its codomain.
In the example above,
  both the domain and codomain are the set of integers~$\Int$,
  but the image is $\Set{x^2 \mid x\in\Int}$,
    which is clearly a subset of~$\Int$.

We therefore have

\begin{align}
  A \rightarrow B &\subset A\times B, \\
  f : A \rightarrow B &\infer f \subset A\times B, \\
  f = \alpha \mapsto \alpha' : A \rightarrow B
                  &= \Set{(\alpha,\alpha')
                            \mid \alpha\in A \land \alpha'\in B}, \\
  f[D\subseteq A] &= \Set{f(\alpha) \mid \alpha\in D} \subset B, \\
  f[] &= f[A].
\end{align}

\indexsym{()}{tuple}
\index{tuple (\ensuremath{()})}
\index{relation|see {function}}
An ordered pair $(x,y)$ is also called a \dfn{$2$-tuple}.
Generally,
  an \dfn{$n$-tuple} is used to represent an $n$-ary function,
    where by convention we have $(x)=x$.
So $f(x,y) = f((x,y)) = x+y$.
If we let $t=(x,y)$,
  then we also have $f(x,y) = ft$,
    which we'll sometimes write as a subscript~$f_t$ where disambiguation is
      necessary and where parenthesis may add too much noise;
        this notation is especially well-suited to indexes,
          as in $f_1$.
Binary functions are often written using \dfn{infix} notation;
  for example, we have $x+y$ rather than $+(x,y)$.

\begin{equation}
  f_x = f(x) \in \Set{b \mid (x,b) \in f}
\end{equation}


\subsubsection{Binary Operations On Functions}
\indexsym{R}{relation}
Consider two unary functions $f$ and~$g$,
  and a binary relation~$R$.
\indexsym{\bicomp{R}}{function, binary composition}
\index{function!binary composition (\ensuremath{\bicomp{R}})}
We introduce a notation~$\bicomp R$ to denote the composition of a binary
  function with two unary functions.

\begin{align}
  f &: A \rightarrow B \\
  g &: A \rightarrow D \\
  R &: B\times D \rightarrow F \\
  f \bicomp{R} g &= \alpha \mapsto f_\alpha R g_\alpha : A \rightarrow F
\end{align}

\indexsym\circ{function, composition}
\index{function!composition (\ensuremath\circ)}
Note that $f$ and~$g$ must share the same domain~$A$.
In that sense,
  this is the mapping of the operation~$R$ over the domain~$A$.
This is analogous to unary function composition~$f\circ g$.

\index{function!constant}
A scalar value~$x$ can be mapped onto some function~$f$ using a constant
  function.
For example,
  consider adding some number~$x$ to each element in the image of~$f$:

\begin{equation*}
  f \bicomp+ (\_\mapsto x) = \alpha \mapsto f_\alpha + x.
\end{equation*}

\indexsym{\_}{variable, wildcard}
\index{variable!wildcard/hole (\ensuremath{\_})}
The symbol~$\_$ is used to denote a variable that matches anything but is
  never referenced,
    and is often referred to as a ``wildcard'' (since it matches anything)
    or a ``hole'' (since its value goes nowhere).

Note that we consider the bracket notation for the image of a function
  $(f:A\rightarrow B)[A]$ to itself be a binary function.
Given that, we have $f\bicomp{[]} = f\bicomp{[A]}$ for functions returning
  functions (such as vectors of vectors in \secref{vec}).


\subsection{Monoids and Sequences}\seclabel{monoids}
\index{abstract algebra!monoid}
\index{monoid|see abstract algebra, monoid}
\begin{definition}[Monoid]\dfnlabel{monoid}
  Let $S$ be some set.  A \dfn{monoid} is a triple $\Monoid S\bullet e$
    with the axioms

  \begin{align}
    \bullet &: S\times S \rightarrow S
      \tag{Monoid Binary Closure} \\
    \Forall{a,b,c\in S&}{a\bullet(b\bullet c) = (a\bullet b)\bullet c)},
      \tag{Monoid Associativity} \\
    \Exists{e\in S&}{\Forall{a\in S}{e\bullet a = a\bullet e = a}}.
      \tag{Monoid Identity}\label{eq:monoid-identity}
  \end{align}
\end{definition}

\index{abstract algebra}
\index{abstract algebra!semigroup}
Monoids originate from abstract algebra.
A monoid is a semigroup with an added identity element~$e$.
Only the identity element must be commutative,
  but if the binary operation~$\bullet$ is \emph{also} commutative,
    then the monoid is a \dfn{commutative monoid}.\footnote{%
      A commutative monoid is less frequently referred to as an
        \dfn{abelian monoid},
          related to the common term \dfn{abelian group}.}

Consider some sequence of operations
  $x_0 \bullet\cdots\bullet x_n \in S$.
Intuitively,
  a monoid tells us how to combine that sequence into a single element
  of~$S$.
When the sequence has one or zero elements,
  we then use the identity element $e\in S$:
    as $x_0 \bullet e = x_0$ in the case of one element
    or $e \bullet e = e$ in the case of zero.

\indexsym\cdots{sequence}
\index{sequence}
\begin{definition}[Monoidic Sequence]\dfnlabel{monoid-seq}
Generally,
  given some monoid $\Monoid S\bullet e$ and a sequence $\Fam{x}jJ\in S$
  where $n<|J|$,
    we have
    $x_0\bullet x_1\bullet\cdots\bullet x_{n-1}\bullet x_n$
    represent the successive binary operation on all indexed elements
    of~$x$.
When it's clear from context that the index is increasing by a constant
  of~$1$,
    that notation is shortened to $x_0\bullet\cdots\bullet x_n$ to save
    space.
When $|J|=1$, then $n=0$ and we have the sequence $x_0$.
When $|J|=0$, then $n=-1$,
  and no such sequence exists,
  in which case we expand into the identity element~$e$.
\end{definition}

For example,
  given the monoid~$\Monoid\Int+0$,
  the sequence $1+2+\cdots+4+5$ can be shortened to
    $1+\cdots+5$ and represents the arithmetic progression
    $1+2+3+4+5=15$.
If $x=\Set{1,2,3,4,5}$,
  $x_0+\cdots+x_n$ represents the same sequence.
If $x=\Set{1}$,
  that sequence evaluates to $1=1$.
If $x=\Set{}$,
  we have $0$.

\index{conjunction!monoid}
\begin{lemma}\lemlabel{monoid-land}
  $\Monoid\Bool\land\true$ is a commutative monoid.
\end{lemma}
\begin{proof}
  $\Monoid\Bool\land\true$ is associative by \dfnref{conj-assoc}
    and commutative by \dfnref{conj-commut}.
  The identity element is~$\true\in\Bool$ by \dfnref{conj-simpl}.
\end{proof}

\index{disjunction!monoid}
\begin{lemma}\lemlabel{monoid-lor}
  $\Monoid\Bool\lor\false$ is a commutative monoid.
\end{lemma}
\begin{proof}
  $\Monoid\Bool\lor\false$ is associative by \dfnref{disj-assoc}
    and commutative by \dfnref{disj-commut}.
  The identity $\false\in\Bool$ follows from

  \begin{alignat*}{3}
    \false \lor p &\equiv p \lor \false &&\text{by \dfnref{disj-commut}} \\
                  &\equiv \neg(\neg p \land \neg\false)\qquad
                    &&\text{by \dfnref{demorgan}} \\
                  &\equiv \neg(\neg p) &&\text{by \dfnref{conj-simpl}} \\
                  &\equiv p. &&\text{by \dfnref{double-neg}} \tag*\qedhere
  \end{alignat*}
\end{proof}


\goodbreak% Fits well on its own page, if we're near a page boundary
\subsection{Vectors and Index Sets}\seclabel{vec}
\tame{} supports scalar, vector, and matrix values.
Unfortunately,
  its implementation history leaves those concepts a bit tortured.

A vector is a sequence of values, defined as a function of
  an index~set.

% TODO: font changes in index, making langle unavailable
%\indexsym{\Vector{}}{vector}
\index{vector!definition (\ensuremath{\Vector{}})}
\index{sequence|see vector}
\indexsym\Vectors{vector}
\index{real number (\ensuremath\Real)}
\indexsym\Real{real number}
\indexsym{\Fam{a}jJ}{index set}
\index{family|see {index set}}
\index{index set (\ensuremath{\Fam{a}jJ})}
\begin{definition}[Vector]\dfnlabel{vec}
  Let $J\subset\Int$ represent an index set.
  A \dfn{vector}~$v\in\Vectors^\Real$ is a totally ordered sequence of
   elements represented as a function of an element of its index set:
  \begin{equation}\label{vec}
    v = \Vector{v_0,\ldots,v_j}^{\Real}_{j\in J}
      = j \mapsto v_j : J \rightarrow \Real.
  \end{equation}
\end{definition}

This definition means that $v_j = v(j)$,
  making the subscript a notational convenience.
We may omit the superscript such that $\Vectors^\Real=\Vectors$
  and $\Vector{\ldots}^\Real=\Vector{\ldots}$.

\index{vector!matrix}
\begin{definition}[Matrix]\dfnlabel{matrix}
  Let $J\subset\Int$ represent an index set.
  A \dfn{matrix}~$M\in\Matrices$ is a totally ordered sequence of
   elements represented as a function of an element of its index set:
  \begin{equation}
    M = \Vector{M_0,\ldots,M_j}^{\Vectors^\Real}_{j\in J}
      = j \mapsto M_j : J \rightarrow \Vectors^\Real.
  \end{equation}
\end{definition}

The consequences of \dfnref{matrix}---%
  defining a matrix as a vector of independent vectors---%
  are important.
This defines a matrix to be more like a multidimensional array,
  with no requirement that the lengths of the vectors be equal.

\begin{corollary}[Matrix Row Length Variance]\corlabel{matrix-row-len}
  $\infer \Exists{M\in\Matrices}{\neg\Forall*{j}{\Forall{k}{\len{M_j} = \len{M_k}}}}$.
\end{corollary}

\corref{matrix-row-len} can be read ``there exists some matrix~$M$ such that
  not all row lengths of~$M$ are equal''.
In other words---%
  the inner vectors of a matrix can vary in length.

Since a vector is a function,
  a vector or matrix can be converted into a set of unique elements like so:

\begin{alignat*}{2}
  \bigcup\Vector{\Vector{0,1},\Vector{2,2},\Vector{2,0}}\!\bicomp{[]}
    &\mapsto &&\bigcup\Vector{\Vector{0,1}\![],\Vector{2,2}\![],\Vector{2,0}[]}\![] \\
    &\mapsto &&\bigcup\Vector{\Set{0,1},\Set{2},\Set{2,0}}\![] \\
    &\mapsto &&\bigcup\Set{\Set{0,1},\Set{2},\Set{2,0}} \\
    &=       &&\Set{0,1,2}.
\end{alignat*}


\subsection{XML Notation}
\indexsym{\xml{<>}}{XML}
\index{XML!notation (\xml{<>})}
The grammar of \tame{} is XML.
Equivalence relations will be used to map source expressions to an
  underlying mathematical expression.
For example,

\begin{equation*}
  \xml{<match on="$x$" value="$y$" />} \equiv x = y
\end{equation*}

\noindent
defines that pattern of \xmlnode{match} expression to be materially
  equivalent to~$x=y$---%
    anywhere an equality relation appears,
      you could equivalently replace it with that XML representation without
        changing the meaning of the mathematical expression.
