
\section{Notational Conventions}
This section provides a fairly terse overview of the foundational
  mathematical concepts used in this paper.
While we try to reason about \tame{} in terms of algebra,
  first-order logic;
  and set theory;
    notation varies even within those branches.
To avoid ambiguity,
  especially while introducing our own notation,
  core operators and concepts are explicitly defined below.

This section begins its numbering at~0.
This is not only a hint that \tame{} (and this paper) use 0-indexing,
  but also because equations; definitions; theorems; corollaries; and the
  like are all numbered relative to their section.
When you see any of these prefixed with ``0.'',
  this sets those references aside as foundational mathematical concepts
    that are not part of the theory and operation of \tame{} itself.


\subsection{Propositional Logic}
\index{boolean!false@\false{}}%
\index{boolean!true@\true{}}%
\index{boolean!FALSE@\tamefalse{}}%
\index{boolean!TRUE@\tametrue{}}%
\index{integer (\Int)}%
We reproduce here certain axioms and corollaries of propositional logic for
  convenience and to clarify our interpretation of certain concepts.
The use of the symbols $\logand$, $\logor$, and~$\neg$ are standard.
The symbol $\vdash$ means ``infer''.
We use $\implies$ in place of $\rightarrow$ for implication,
  since the latter is used to denote the mapping of a domain to a codomain
  in reference to functions.
We further use $\equiv$ in place of $\leftrightarrow$ to represent material
  equivalence.

\begin{definition}[Logical Conjunction]
  $p,q \vdash (p\logand q)$.
\end{definition}

\begin{definition}[Logical Disjunction]
  $p \vdash (p\logor q)$ and $q \vdash (p\logor q)$.
\end{definition}

\begin{definition}[Law of Excluded Middle]
  $\vdash (p \logor \neg p)$.
\end{definition}

\begin{definition}[Law of Non-Contradiction]
  $\vdash \neg(p \logand \neg p)$.
\end{definition}

\begin{definition}[De Morgan's Theorem]
  $\neg(p \logand q) \vdash (\neg p \logor \neg q)$
    and $\neg(p \logor q) \vdash (\neg p \logand \neg q)$.
\end{definition}

\index{equivalence!material (\ensuremath{\equiv})}
\begin{definition}[Material Equivalence]
  $p\equiv q \vdash \big((p \logand q) \logor (\neg p \logand \neg q)\big)$.
\end{definition}

$\equiv$ denotes a logical identity.
Consequently,
  it'll often be used as a definition operator.

\begin{definition}[Logical Implication]
  $p\implies q \vdash (\neg p \logor q)$.
\end{definition}

\begin{definition}[Truth Values]\dfnlabel{truth-values}
  $\vdash\true$ and $\vdash\neg\false$.
\end{definition}


\subsection{First-Order Logic and Set Theory}
The symbol $\emptyset$ represents the empty set---%
  the set of zero elements.
We assume that the axioms of ZFC~set theory hold,
  but define $\in$ here for clarity.

\index{set!membership@membership (\ensuremath{\in})}
\begin{definition}[Set Membership]
  $x \in S \equiv \Set{x} \cap S \not= \emptyset.$
\end{definition}

\index{quantification|see {fist-order logic}}
\index{first-order logic!quantification (\ensuremath{\forall, \exists})}
$\forall$ denotes first-order universal quantification (``for all''),
  and $\exists$ first-order existential quantification (``there exists''),
  over some \gls{domain}.

\index{disjunction|see {first-order logic}}
\index{first-order logic!disjunction (\ensuremath{\logor})}
\begin{definition}[Existential Quantification]\dfnlabel{exists}
  $\Exists{x\in X}{P(x)} \equiv
    \true \in \Set{P(x) \mid x\in X}$.
\end{definition}

\index{conjunction|see {first-order logic}}
\index{first-order logic!conjunction (\ensuremath{\logand})}
\begin{definition}[Universal Quantification]\dfnlabel{forall}
  $\Forall{x\in X}{P(x)} \equiv \neg\Exists{x\in X}{\neg P(x)}$.
\end{definition}

\index{set!empty (\ensuremath{\emptyset, \{\}})}
\begin{remark}[Vacuous Truth]
  By Definition~7, $\Exists{x\in\emptyset}P \equiv \false$
    and by \dfnref{forall}, $\Forall{x\in\emptyset}P \equiv \true$.
  And so we also have the tautologies $\vdash \neg\Exists{x\in\emptyset}P$
    and $\vdash \Forall{x\in\emptyset}P$.
\end{remark}

\begin{definition}[Boolean/Integer Equivalency]\dfnlabel{bool-int}
  $\Set{0,1}\in\Int, \false \equiv 0$ and $\true \equiv 1$.
\end{definition}

\tamefalse{} and~\tametrue{} are constants in \tame{} mapping to the
  \gls{integer} values $\{0,1\}\in\Int$.
\dfnref{bool-int} relates these constants to their
  \gls{boolean} counterparts so that they may be used in numeric contexts
  and vice-versa.


\subsection{Functions}
The notation $f = x \mapsto x' : A\rightarrow B$ represents a function~$f$
  that maps from~$x$ to~$x'$,
    where $x\in A$ (the domain of~$f$) and $x'\in B$ (the co-domain of~$f$).

A function $A\rightarrow B$ can be represented as the Cartesian
  product of its domain and codomain, $A\times B$.
For example,
  $x\mapsto x^2 : \Int\rightarrow\Int$ is represented by the set of ordered
  pairs $\Set{(x,x^2) \mid x\in\Int}$, which looks something like

\begin{equation*}
  \Set{\ldots,\,(0,0),\,(1,1),\,(2,4),\,(3,9),\,\ldots}.
\end{equation*}

The set of values over which some function~$f$ ranges is its \emph{image},
  which is a subset of its codomain.
In the example above,
  both the domain and codomain are the set of integers~$\Int$,
  but the image is $\Set{x^2 \mid x\in\Int}$,
    which is clearly a subset of~$\Int$.

We therefore have

\begin{align}
  A \rightarrow B &\subset A\times B, \\
  f : A \rightarrow B &\vdash f \subset A\times B, \\
  f = \alpha \mapsto \alpha' : A \rightarrow B
                  &= \Set{(\alpha,\alpha')
                            \mid \alpha\in A \logand \alpha'\in B}, \\
  f[D\subseteq A] &= \Set{f(\alpha) \mid \alpha\in D} \subset B, \\
  f[] &= f[A].
\end{align}

And ordered pair $(x,y)$ is also called a \emph{$2$-tuple}.
Generally,
  an \emph{$n$-tuple} is used to represent an $n$-ary function,
    where by convention we have $(x)=x$.
So $f(x,y) = f((x,y)) = x+y$.
If we let $t=(x,y)$,
  then we also have $f(x,y) = ft$.
Binary functions are often written using \emph{infix} notation;
  for example, we have $x+y$ rather than $+(x,y)$.

\begin{equation}
  fx \in \Set{b \mid (x,b) \in f}
\end{equation}


\subsubsection{Binary Operations On Functions}
Consider two unary functions $f$ and~$g$,
  and a binary relation~$R$.
We introduce a notation~$\bicomp R$ to denote the composition of a binary
  function with two unary functions.\footnote{%
    The notation originates from~$\circ$ to denote ordinary function
      composition,
        as in $(f\circ g)(x) = f(g(x))$.}

\begin{align}
  f &: A \rightarrow B \\
  g &: A \rightarrow D \\
  R &: B\times D \rightarrow F \\
  f \bicomp{R} g &= \alpha \mapsto f(\alpha)Rg(\alpha) : A \rightarrow F
\end{align}

Note that $f$ and~$g$ must share the same domain~$A$.
In that sense,
  this is the mapping of the operation~$R$ over the domain~$A$.
This is analogous to unary function composition~$f\circ g$.

A scalar value~$x$ can be mapped onto some function~$f$ using a constant
  function.
For example,
  consider adding some number~$x$ to each element in the image of~$f$:

\begin{equation*}
  f \bicomp+ (\_\mapsto x) = \alpha \mapsto f(\alpha) + x.
\end{equation*}

The symbol~$\_$ is used to denote a variable that is never referenced.

For convenience,
  we also define $\bicompi{R}$,
  which recursively handles combinations of function and scalar values.
This notation is used to simplify definitions of the classification system
  (see \secpref{class})
  when dealing with vectors
    (see \secref{vec}).

\begin{equation}\label{eq:bicompi}
  \alpha \bicompi{R} \beta =
    \begin{cases}
      \gamma \mapsto \alpha(\gamma) \bicompi{R} \beta(\gamma)
        &\text{if } (\alpha : A\rightarrow B) \logand (\beta : A\rightarrow D),\\
      \gamma \mapsto \alpha(\gamma) \bicompi{R} (\_ \mapsto \beta)
        &\text{if } (\alpha : A\rightarrow B) \logand (\beta \in\Real),\\
      \alpha R \beta &\text{otherwise}.
    \end{cases}
\end{equation}

Note that we consider the bracket notation for the image of a function
  $(f:A\rightarrow B)[A]$ to itself be a binary function.
Given that, we have $f\bicomp{[]} = f\bicomp{[A]}$ for functions returning
  functions (such as vectors of vectors in \secref{vec}),
    noting that $\bicompi{[]}$ is \emph{not} a sensible construction.


\subsection{Vectors and Index Sets}\seclabel{vec}
\tame{} supports scalar, vector, and matrix values.
Unfortunately,
  its implementation history leaves those concepts a bit tortured.

A vector is a sequence of values, defined as a function of
  an~\gls{index set}.

\begin{definition}[Vector]\dfnlabel{vec}
  Let $J\subset\Int$ represent an index set.
  A \emph{vector}~$v\in\Vectors^\Real$ is a totally ordered sequence of
   elements represented as a function of an element of its index set:
  \begin{equation}\label{vec}
    v = \Vector{v_0,\ldots,v_j}^{\Real}_{j\in J}
      = j \mapsto v_j : J \rightarrow \Real.
  \end{equation}
\end{definition}

This definition means that $v_j = v(j)$,
  making the subscript a notational convenience.
We may omit the superscript such that $\Vectors^\Real=\Vectors$
  and $\Vector{\ldots}^\Real=\Vector{\ldots}$.

\begin{definition}[Matrix]\dfnlabel{matrix}
  Let $J\subset\Int$ represent an index set.
  A \emph{matrix}~$M\in\Matrices$ is a totally ordered sequence of
   elements represented as a function of an element of its index set:
  \begin{equation}
    M = \Vector{M_0,\ldots,M_j}^{\Vectors^\Real}_{j\in J}
      = j \mapsto M_j : J \rightarrow \Vectors^\Real.
  \end{equation}
\end{definition}

The consequences of \dfnref{matrix}---%
  defining a matrix as a vector of independent vectors---%
  are important.
This defines a matrix to be more like a multidimensional array,
  with no requirement that the lengths of the vectors be equal.

\begin{corollary}[Matrix Row Length Variance]\corlabel{matrix-row-len}
  $\vdash \Exists{M\in\Matrices}{\neg\Forall*{j}{\Forall{k}{\len{M_j} = \len{M_k}}}}$.
\end{corollary}

\corref{matrix-row-len} can be read ``there exists some matrix~$M$ such that
  not all row lengths of~$M$ are equal''.
In other words---%
  the inner vectors of a matrix can vary in length.

Since a vector is a function,
  a vector or matrix can be converted into a set of unique elements like so:

\begin{alignat*}{2}
  \bigcup\Vector{\Vector{0,1},\Vector{2,2},\Vector{2,0}}\!\bicomp{[]}
    &\mapsto &&\bigcup\Vector{\Vector{0,1}\![],\Vector{2,2}\![],\Vector{2,0}[]}\![] \\
    &\mapsto &&\bigcup\Vector{\Set{0,1},\Set{2},\Set{2,0}}\![] \\
    &\mapsto &&\bigcup\Set{\Set{0,1},\Set{2},\Set{2,0}} \\
    &=       &&\Set{0,1,2}.
\end{alignat*}

We can also add two vectors, and scale them:

\begin{align*}
  1 \bicomp{+} \Vector{1,2,3} \bicomp{+} \Vector{4,5,6}
    &= \Vector{1+1,\, 2+1,\, 3+1} \bicomp{+} \Vector{4,5,6} \\
    &= \Vector{2,3,4} \bicomp{+} \Vector{4,5,6} \\
    &= \Vector{2+4,\, 3+5,\, 4+6} \\
    &= \Vector{6, 8, 10}.
\end{align*}


\subsection{XML Notation}
\index{XML}
The grammar of \tame{} is XML.
Equivalence relations will be used to map source expressions to an
  underlying mathematical expression.
For example,

\begin{equation*}
  \xml{<match on="$x$" value="$y$" />} \equiv x = y
\end{equation*}

\noindent
defines that pattern of \xmlnode{match} expression to be materially
  equivalent to~$x=y$---%
    anywhere an equality relation appears,
      you could equivalently replace it with that XML representation without
        changing the meaning of the mathematical expression.
